\documentclass{article}

\title{RSA-OAEP IND-CCA2 증명}
\author{김동현(wlswudpdlf31@kookmin.ac.kr)}
\date{\today}

\usepackage{style}

% 한국어 지원 패키지
\usepackage{kotex}

% Document setting
\usepackage{geometry}
\geometry{
	a4paper, 
	left=3cm, right=3cm, top=2cm, bottom=2cm, 
	includehead, includefoot}
	\usepackage{fancyhdr} % 머리말과 꼬리말 설정 
\usepackage{lastpage}
\pagestyle{fancy}
\fancyhf{} % 기존 머리말/꼬리말 초기화
\renewcommand{\headrulewidth}{0.4pt} % 머리말 선 두께
\fancyhead[L]{\leftmark} % 왼쪽 머리말
\fancyhead[R]{\rightmark} % 오른쪽 머리말
\renewcommand{\footrulewidth}{0.4pt}
\fancyfoot[L]{FDL}
\fancyfoot[R]{\thepage~/~\pageref{LastPage}} % 가운데 꼬리말에 페이지 번호 추가


% 수학 관련 패키지
\usepackage{amsmath, amssymb}
\usepackage{amsthm}
\usepackage{mathtools} % mathtools 패키지 필요
\newtheorem{theorem}{정리}
\newtheorem{lemma}{보조정리}
\theoremstyle{definition}
\newtheorem{memo}{메모}

% 이미지 관련 패키지
\usepackage{graphicx}

\usepackage{hyperref}

% TikZ 패키지 추가
\usepackage{tikz}

% Table package and setting
\usepackage{tcolorbox}
\usepackage{tabularx}
\usepackage{colortbl} % colortbl 패키지 추가
\newcolumntype{C}{>{\centering\arraybackslash}X}


\begin{document}
\maketitle
\tableofcontents

\newpage
\section{논문정보}

\begin{itemize}
	\item 제목: RSA-OAEP is Secure under the RSA Assumption
	\item 저자: Eiichiro Fujisaki1, Tatsuaki Okamoto, David Pointcheval, and Jacques Stern
	\item 년도: 2001년
	\item 초록: 최근 Victor Shoup은 적응적 선택 암호문 공격에 대한 OAEP의 보안성에 관한 널리
	받아들여진 결과에 틈이 있음을 지적하였다. 더욱이, 그는 기본 트랩도어 치환의
	단방향성만으로는 OAEP의 보안성을 증명할 수 없을 것으로 예상된다는 점을 보였다.
	본 논문은 OAEP의 보안성에 대한 또 다른 결과를 제시한다. 즉, 본 논문에서는 무작위
	오라클 모델에서, 기본 치환의 부분 영역 단방향성(partial-domain one-wayness)
	하에서, OAEP가 적응적 선택 암호문 공격에 대해 의미론적 보안성을 제공함을
	증명한다. 따라서, 이는 형식적으로 더 강한 가정을 사용한다. 그럼에도 불구하고,
	RSA 함수의 부분 영역 단방향성이 (전체 영역) 단방향성과 동치이므로, RSA-OAEP의
	보안성은 단순한 RSA 가정만으로도 증명될 수 있음을 알 수 있다. 다만, 그
	축소(reduction)는 타이트하지 않다.
\end{itemize}

% \section{기호}

% \begin{itemize}
% 	\item $\sp$ 보안 매개변수
% 	\item $\pk$ 공개키
% 	\item $\sk$ 비밀키
% 	\item $(\pk, \sk) \gets \genkey(1^{\sp})$ 키 생성 알고리즘, 확률론적
% 	\item $\msg$ 메시지
% 	\item $\ct$ 암호문
% 	\item $\ct \gets \enc(\msg, \pk)$ 암호화 함수, 확률론적
% 	\item $\msg \gets \dec(\ct, \sk)$ 복호화 함수, 결정론적
% 	\item $\adv$ 공격자
% 	\item $\coin$ 랜덤 코인
% 	\item $\msgspace$ 메시지 공간
% 	\item $\coinspace$ 랜덤 코인 공간
% 	\item $\bit$ 0 또는 1
% 	\item $\hash, \gash$ 암호학적 해시 함수
% 	\item $\mlen$ 메시지 길이
% \end{itemize}

\newpage
\section{보안 개념}

% \subsection{$\owcpa$}

% \begin{tcolorbox}[colback=white]
% 	\centering
% 	\begin{tabularx}{\linewidth}{CcC}
% 		\underline{Challenger $\ch$} & $\xLeftrightarrow{\Exp^{\owcpa}_{\sch, \sp}}$ & \underline{Adversary $\adv$} \\
% 		\\
% 		$(\pk, \sk) \gets \genkey(1^{\sp})$ & $\xrightarrow{1^{\sp}, \pk}$ & \\
% 		\\
% 		$\msg^* \rgets \msgspace$ \\ $\ct^* \gets \enc_{\pk}(\msg^*)$ & $\xrightarrow{\ct^*}$ & \\
% 		\\
% 		Return $[\msg' \same \msg^*]$ & $\xleftarrow{\msg'}$ & $\adv$ chooses $\msg' \in \msgspace$ \\
%   \end{tabularx}
% \end{tcolorbox}

% \begin{align}
% 	\begin{split}
% 		\Adv^{\owcpa}_{\adv, \sch}(\sp)
% 		&= \Pr[\Exp^{\owcpa}_{\sch, \sp}(\adv) = 1] \\
% 		&= \underset{\msg^*}{\Pr}[
% 				(\pk, \sk) \gets \genkey(1^{\sp}): 
% 				\adv(\pk, \enc_{\pk}(m^*)) = \msg^*].
% 	\end{split}
% \end{align}

\subsection{OW trapdoor permutation}
트랩도어 치환 체계(Trapdoor permutation scheme) $\tdsch = (\genkey, \td, \itd)$를 다음과
같이 정의한다.
\begin{itemize}
	\item $\genkey(1^{\sp})$: 확률론적 키 생성 알고리즘으로, $1^{\sp}$를 입력
	받아 $(\pk, \sk)$를 생성한다.
	\item $\td_{\pk}(x)$: 결정론적 알고리즘으로, $\pk$와 $x \in \set{0, 1}^{\sp}$를 입력 받아
	$y \in \set{0 ,1}^{\sp}$를 출력한다.
	\item $\itd_{\sk}(y)$: 결정론적 알고리즘으로, $\sk$와 $y \in \set{0,
	1}^{\sp}$를 입력 받아 $x \in \set{0, 1}^{\sp}$를 출력한다.
	$\genkey(1^{\sp})$로 생성한 모든 $(\pk, \sk)$와 모든 $x \in \set{0,
	1}^{\sp}$에 대해, $\itd_{\sk}(\td_{\pk}(x)) = x$를 만족한다.
\end{itemize}
동작시간(Running time) $\rt$를 가지는 공격자 $\adv$와 트랩도어 치환 체계
$\tdsch$에 대한 일방향성(One-wayness) 실험 $\Exp^{\ow}_{\tdsch, \sp}(\adv; \rt)$을
다음과 같이 정의한다.
\begin{tcolorbox}[colback=white]
	\centering
	\begin{tabularx}{\linewidth}{CcC}
		\underline{Challenger $\ch$} & $\xLeftrightarrow{\Exp^{\ow}_{\tdsch, \sp}}$ & \underline{Adversary $\adv$} \\
		\\
		$(\pk, \sk) \rgets \genkey(1^{\sp})$ & & \\
		\\
		$x^* \rgets \set{0, 1}^{\sp}$ \\ $y^* \gets \td_{\pk}(x^*)$ & $\xrightarrow{1^{\sp}, \pk, y^*}$ & \\
		\\ 
		Return $[x' \same x^*]$ & $\xleftarrow{x'}$ & $\adv$ chooses $x' \in \xspace$ \\
  \end{tabularx}
\end{tcolorbox}

$\adv$의 능력치 $\Adv^{\ow}_{\adv;
\tdsch}(\sp, \rt)$를 다음과 같이 정의한다.
$$
	\Adv^{\ow}_{\tdsch, \sp}(\adv; \rt) := \Pr[\Exp^{\ow}_{\tdsch, \sp}(\adv; \rt) = 1].
$$
% 공격자 $\adv$의 능력치가 어떤 $\negl$에 대해 다음을 만족할 때, $\tdsch$가 $(\rt,
% \negl)$-일방향성을 가진다고 한다.
% $$
% 	\Adv^{\ow}_{\adv, \tdsch}(\sp, \rt) \le \negl.
% $$

\subsection{Partial-domain OW trapdoor permutation}

트랩도어 치환 $\tdsch = (\genkey, \td, \itd)$에서, $\td_{\pk}(x):\set{0, 1}^{\sp} \to
\set{0, 1}^{\sp}$를 다음과 같이 표현한다.
$$
	\td_{\pk}: \set{0, 1}^{\mlen + \sp_1} \times \set{0, 1}^{\sp_0} \to \set{0, 1}^{\mlen + \sp_1} \times \set{0, 1}^{\sp_0}.
$$
이때 $\sp = \mlen + \sp_0 + \sp_1$이다. 
\begin{memo}
	예를 들어, $x = s \parallel t$라고 할 때, $y \gets \td_{\pk}(x)$ 대신 $y
	\gets \td_{\pk}(s \parallel t)$로 표현할 수 있다.
\end{memo}
동작시간 $\rt$를 가지는 공격자 $\adv$와 트랩도어 함수 체계 $\tdsch$에 대한 부분
일방향성(Partial-domain one-wayness) 실험 $\Exp^{\owpd}_{\tdsch, \sp}(\adv;
\rt)$을 다음과 같이 정의한다.

\begin{tcolorbox}[colback=white]
	\centering
	\begin{tabularx}{\linewidth}{CcC}
		\underline{Challenger $\ch$} & $\xLeftrightarrow{\Exp^{\owpd}_{\tdsch, \sp}}$ & \underline{Adversary $\adv$} \\
		\\
		$(\pk, \sk) \rgets \genkey(1^{\sp})$ & & \\
		\\
		$(s^*, t^*) \rgets \set{0, 1}^{\mlen + \sp_1} \times \set{0, 1}^{\sp_0}$ \\ $y^* \gets \td_{\pk}(s^*, t^*)$ & $\xrightarrow{1^{\sp}, \pk, y^*}$ & \\
		\\
		Return $[s' \same s^*]$ & $\xleftarrow{s'}$ & $\adv$ chooses $s' \in \set{0, 1}^{\mlen + \sp_1}$ \\
  \end{tabularx}
\end{tcolorbox}

공격자 $\adv$의 능력치 $\Adv^{\owpd}_{\tdsch, \sp}(\adv; \rt)$를 다음과 같이 정의한다.
$$
	\Adv^{\owpd}_{\tdsch, \sp}(\adv; \rt) := \Pr[\Exp^{\owpd}_{\tdsch, \sp}(\adv; \rt) = 1].
$$
% 동작시간 $\rt$를 가지는 공격자 $\adv$의 능력치가 어떤 $\negl$에 대해 다음을
% 만족할 때, $\tdsch$가 $(\rt, \negl)$-부분 일방향성을 가진다고 한다.
% $$
% 	\Adv^{\owpd}_{\adv, \tdsch}(\sp, \rt) \le \negl.
% $$

\subsection{Set partial-domain OW trapdoor permutation}

동작시간 $\rt$를 가지고 $l$개의 원소를 출력하는 공격자 $\adv$와 트랩도어 함수
체계 $\tdsch$에 대한 집합 부분 일방향성(Set partial-domain one-wayness) 실험
$\Exp^{\owpds}_{\tdsch, \sp}(\adv; \rt, l)$을 다음과 같이 정의한다.
\begin{tcolorbox}[colback=white]
	\centering
	\begin{tabularx}{\linewidth}{CcC}
		\underline{Challenger $\ch$} & $\xLeftrightarrow{\Exp^{\owpds}_{\tdsch, \sp}}$ & \underline{Adversary $\adv$} \\
		\\
		$(\pk, \sk) \rgets \genkey(1^{\sp})$ & & \\
		\\
		$(s^*, t^*) \rgets \set{0, 1}^{\mlen + \sp_1} \times \set{0, 1}^{\sp_0}$ \\ $y^* \gets \td_{\pk}(s^*, t^*)$ & $\xrightarrow{1^{\sp}, \pk, y^*}$ & \\
		\\
		Return $[s^* \isin S']$ & $\xleftarrow{S'}$ & $\adv$ chooses $S' = \set{s_i \in \set{0, 1}^{\mlen + \sp_1}: 0 \le i \le l-1}$ \\
  \end{tabularx}
\end{tcolorbox}

공격자 $\adv$의 능력치 $\Adv^{\owpds}_{\tdsch, \sp}(\adv; \rt, l)$를 다음과 같이 정의한다.
$$
	\Adv^{\owpds}_{\tdsch, \sp}(\adv; \rt, l) := \Pr[\Exp^{\owpds}_{\tdsch, \sp}(\adv; \rt, l) = 1].
$$
% 동작시간 $\rt$를 가지는 공격자 $\adv$의 능력치가 어떤 $\negl$에 대해 다음을
% 만족할 때, $\tdsch$가 $(l, \rt, \negl)$-집합 부분 일방향성을 가진다고 한다.
% $$
% 	\Adv^{\owpds}_{\adv, \tdsch}(\sp, \rt, l) \le \negl.
% $$

\subsection{IND security against CCA2}

공개키 암호
체계(Public-key encryption scheme) $\sch = (\genkey, \enc, \dec)$를 다음과 같이
정의한다.
\begin{itemize}
	\item $\genkey(1^{\sp})$: 확률론적 키 생성 알고리즘으로, $1^{\sp}$를 입력
	받아 $(\pk, \sk)$를 생성한다.
	\item $\enc_{\pk}(\msg)$: 암호화 알고리즘으로, $\pk$와 $\msg \in
	\msgspace$를 입력 받아 $\ct \in \ctspace$를 출력한다. 확률론적 알고리즘으로,
	$\coin \rgets \coinspace$를 추가로 입력 받아 $\enc_{\pk}(\msg; \coin)$으로
	표현할 수도 있다.
	\item $\dec_{\sk}(\ct)$: 결정론적 복호화 알고리즘으로, $\sk$와 $\ct \in
	\ctspace$를 입력 받아 $\msg \in \msgspace$를 출력한다.
\end{itemize}

동작시간 $\rt$를 가지고 복호화 오라클에 $\qr$회 질의하는 공격자 $\adv$와 공개키
암호 체계 $\sch$에 대해, 선택 암호문 공격(Adaptive chosen ciphertext attack,
이하 CCA2)에 대한 구별불가능성(Indistinguishability)
실험 $\Exp^{\indcca}_{\sch, \sp}(\adv; \rt, \qr)$을 다음과 같이 정의한다.

\begin{tcolorbox}[colback=white]
	\centering
	\begin{tabularx}{\linewidth}{CcC}
		\underline{Challenger $\ch$} & $\xLeftrightarrow{\Exp^{\indcca}_{\sch, \sp}}$ & \underline{Adversary $\adv$} \\
		\\
		$(\pk, \sk) \gets \genkey(1^{\sp})$ & $\xrightarrow{1^{\sp}, \pk}$ & \\
		\\
		\multicolumn{3}{c}{\cellcolor{gray!20}$\orc^{\dec_{\sk}}$ Query Phase 1} \\
		\\ 
		& $\xleftarrow{\msg^*_0, \msg^*_1}$ & $\adv$ chooses $\msg^*_0, \msg^*_1 \in \msgspace$ such that $|\msg^*_0| = |\msg^*_1|$ and $\msg^*_0 \neq \msg^*_1$ \\
		\\
		$\bit \rgets \set{0, 1}, \coin^* \rgets \coinspace$ \\ $\ct^* \gets \enc_{\pk}(\msg^*_{\bit}; \coin^*)$ & $\xrightarrow{\ct^*}$ & \\
		\\
		\multicolumn{3}{c}{\cellcolor{gray!20}$\orc^{\dec_{\sk}}$ Query Phase 2} \\
		\\
		Return $[\bit' \same \bit]$ & $\xleftarrow{\bit'}$ & $\adv$ chooses $\bit' \in \set{0, 1}$ \\
  \end{tabularx}
\end{tcolorbox}

공격자 $\adv$의 능력치 $\Adv^{\indcca}_{\sch, \sp}(\adv; \rt, \qr)$를 다음과 같이 정의한다.
$$
	\Adv^{\indcca}_{\sch, \sp}(\adv; \rt, \qr) = 2 \cdot \Pr[\Exp^{\indcca}_{\sch, \sp}(\adv; \rt, \qr) = 1] - 1.
$$

\subsection{IND security against CCA2 in ROM}

동작시간 $\rt$를 가지고  복호화 오라클에 $\qr_{\dec}$회, 랜덤 오라클에
$\qr_{\hash}$회 질의하는 공격자 $\adv$와 공개키 암호 체계 $\sch$에 대해,
랜덤 오라클 모델(Random oracle model)에서의 CCA2에 대한 구별불가능성 실험
$\Exp^{\indccarom}_{\sch, \sp}(\adv; \rt, \qr_{\dec}, \qr_{\hash})$을 다음과
같이 정의한다.

\begin{tcolorbox}[colback=white]
	\centering
	\begin{tabularx}{\linewidth}{CcC}
		\underline{Challenger $\ch$} & $\xLeftrightarrow{\Exp^{\indccarom}_{\sch, \sp}}$ & \underline{Adversary $\adv$} \\
		\\
		$\orc_{\hash} \rgets \hashspace$ \\ $(\pk, \sk) \gets \genkey(1^{\sp})$ & $\xrightarrow{1^{\sp}, \pk}$ & \\
		\\
		\multicolumn{3}{c}{\cellcolor{gray!20}$\orc^{\hash}, \orc^{\dec_{\sk}}$ Query Phase 1} \\
		\\
		& $\xleftarrow{\msg^*_0, \msg^*_1}$ & $\adv$ chooses $\msg^*_0, \msg^*_1 \in \msgspace$ such that $|\msg^*_0| = |\msg^*_1|$ and $\msg^*_0 \neq \msg^*_1$ \\
		\\
		$\bit \rgets \set{0, 1}, \coin^* \rgets \coinspace$ \\ $\ct^* \gets \enc_{\pk}(\msg^*_{\bit}; \coin^*)$ & $\xrightarrow{\ct^*}$ & \\
		\\
		\multicolumn{3}{c}{\cellcolor{gray!20}$\orc^{\hash}, \orc^{\dec_{\sk}}$ Query Phase 2} \\
		\\
		Return $[\bit' \same \bit]$ & $\xleftarrow{\bit'}$ & $\adv$ chooses $\bit' \in \set{0, 1}$ \\
  \end{tabularx}
\end{tcolorbox}

공격자 $\adv$의 능력치 $\Adv^{\indccarom}_{\sch, \sp}(\adv; \rt, \qr_{\dec},
\qr_{\hash})$를 다음과 같이 정의한다.
$$
	\Adv^{\indccarom}_{\sch, \sp}(\adv; \rt, \qr_{\dec},
	\qr_{\hash}) = 2 \cdot \Pr[\Exp^{\indccarom}_{\sch, \sp}(\adv; \rt, \qr_{\dec},
	\qr_{\hash}) = 1] - 1.
$$

\newpage
\section{RSA-OAEP}
다음과 같은 트랩도어 치환 $\td$를 고려한다.
$$
	\td_{\pk}: \set{0, 1}^{\mlen + \sp_1} \times \set{0, 1}^{\sp_0} \to \set{0, 1}^{\mlen + \sp_1} \times \set{0, 1}^{\sp_0}.
$$
그리고 두 해시 함수 $H, G$를 다음과 같이 준비한다.
$$
	\hash: \set{0, 1}^{\sp_0} \to \set{0, 1}^{\sp - \sp_0} \quad
	\gash: \set{0, 1}^{\sp - \sp_0} \to \set{0, 1}^{\sp_0}.
$$
트랩토어 치환 체계 $\tdsch = (\genkey, \td, \itd)$를 포함하는 OAEP 변환
$(\genkey, \enc, \dec)$는 다음과 같이 동작한다.
\begin{itemize}
	\item $\genkey(1^{\sp})$: $(\pk, \sk)$를 생성한다. $\pk$는 이후 트랩도어 치환
	$\td$에서 사용하며, $\sk$는 $\itd$에서 사용한다.
	\item $\enc_{\pk}(\msg; \coin)$: $\msg \in \set{0, 1}^{\mlen}$과 $\coin
	\rgets \set{0, 1}^{\sp_0}$가 주어졌을 때, $s, t$를 다음과 같이 계산한다.
	$$
		s = (\msg \parallel 0^{\sp_1}) \xor \gash(\coin), \quad
		t = \coin \xor \hash(s).
	$$
	$s, t$를 계산하는 과정을 도식화하면 그림 \ref{fig:oaep}와 같다. 이후 암호문
	$\ct = \td_{\pk}(s, t)$를 출력한다.
  	\item $\dec_{\sk}(\ct)$: $(s,t) = \itd_{\sk}(\ct)$을 계산한 후, $r, M$을 다음과
  	같이 계산한다.
	$$
    	r = t \xor \hash(s) \quad M = s \xor \gash(\coin).
	$$
  	만약 $[M]_{\sp_1} = 0^{\sp_1}$이면 $[M]^{\mlen}$을 출력하고, 아니라면
  	“$\textsf{Reject}$”를 출력한다.
\end{itemize}

\begin{figure}[ht]
	\centering
	\begin{tikzpicture}
		\node (m0) at (0, 0) {$\msg \parallel 0^{\sp_0}$};
		\node (r) at (3, 0) {$\coin$};
		\node (g) at (1.5, -1) {$\gash$};
		\node (h) at (1.5, -2) {$\hash$};
		\node (s) at (0, -3) {$s$};
		\node (t) at (3, -3) {$t$};
		\node (xorh) at (0, -1) {$\xor$};
		\node (xorg) at (3, -2) {$\xor$};

		\draw[->] (r) |- (g);
		\draw[->] (r) -- (t);
		\draw[->] (m0) -- (s);
		\draw[->] (m0) |- (h);
		\draw[->] (g) -| (s);
		\draw[->] (h) -| (t);
	\end{tikzpicture}
	\caption{$\enc_{\pk}(\msg; \coin)$에서 $s, t$를 계산하는 과정}
	\label{fig:oaep}
\end{figure}

\section{증명}

\begin{tcolorbox}[colback=white]
	\begin{lemma}
		공격자 $\adv$를 OAEP 변환 $(\genkey, \enc, \dec)$에 대해 동작시간 $\rt$를 가지고,
		복호화 오라클 $\orc^{\dec}$와 랜덤 오라클 $\orc^{\hash}, \orc^{\gash}$에
		각각 $\qr_{\dec}, \qr_{\hash}, \qr_{\gash}$회 질의하는
		$\textsf{IND-CCA2}$ 공격자라 하자. 어떤 $\owpds$ 공격자 $\bdv$에 대해,
		다음을 만족한다.
		$$
			\Adv^{\owpds}_{\tdsch, \sp}(\bdv; \rt', \qr_{\hash})
		  	\ge \frac{\Adv^{\indcca}_{\sch, \sp}(\adv; \rt, \qr_{\dec}, \qr_{\hash}, \qr_{\gash})}{2}
		   	- \frac{2\qr_{\dec}\qr_{\gash} + \qr_{\dec} + \qr_{\gash}}{2^{\sp_0}}
		   	- \frac{2\qr_{\dec}}{2^{\sp_1}}.
		$$
	  이 때, $\rt' \le \rt \cdot \qr_{\hash} \cdot \qr_{\gash} \cdot (\comp{\td} +
	   \bigo(1))$이고, $\comp{\td}$는 트랩도어 치환 $\td$의 시간 복잡도를 의미한다.
	\end{lemma}
\end{tcolorbox}

% \begin{tcolorbox}
% 	우리는 $\text{Succ}^{\text{ow}}(\tau)$, (그리고 각각 $\text{Succ}^{\text{pd-ow}}(\tau)$ 및 $\text{Succ}^{\text{s-pd-ow}}(\ell, \tau)$)를 최대 성공 확률 $\text{Succ}^{\text{ow}}(\mathcal{A})$ (그리고 각각 $\text{Succ}^{\text{pd-ow}}(\mathcal{A})$ 및 $\text{Succ}^{\text{s-pd-ow}}(\mathcal{A})$)로 정의한다. 여기서 최대값은 실행 시간이 $\tau$ 이하로 제한된 모든 공격자(adversary)에 대해 계산된다.

% 세 번째 경우에서는 $\mathcal{A}$가 $\ell$개의 원소를 포함하는 집합을 출력한다는 추가적인 제한이 존재한다. 따라서, 모든 $\tau$ 및 $\ell \geq 1$에 대해 다음이 성립함이 분명하다.

% $$
% \text{Succ}^{\text{s-pd-ow}}(\ell, \tau) \geq \text{Succ}^{\text{pd-ow}}(\tau) \geq \text{Succ}^{\text{ow}}(\tau).
% $$

% 또한, 공격자가 반환한 집합에서 무작위로 원소를 선택함으로써 집합 부분 도메인 단방향성(Set Partial-Domain One-Wayness)을 깨뜨릴 수 있으며, 그 확률은 $\text{Succ}^{\text{s-pd-ow}}(\mathcal{A}) / \ell$이다. 이를 통해 다음의 부등식이 성립한다.

% $$
% \text{Succ}^{\text{pd-ow}}(\tau) \geq \text{Succ}^{\text{s-pd-ow}}(\ell, \tau) / \ell.
% $$

% 그러나 특정한 함수 $f$에 대해 보다 효율적인 환원이 존재할 수도 있다. 또한, 어떤 경우에는 세 가지 문제 모두 다항식적으로 동등할 수 있다. 이는 RSA 치환(RSA permutation)의 경우에 해당하며, 따라서 이에 대한 결과는 6장에서 다룬다.
% \end{tcolorbox}

% 우리는 보조정리 2를 세 단계로 증명한다. 첫 번째 단계에서는 $\indcca$ 적대자
% $\adv$를 부분 도메인 일방성(partial-domain one-wayness) $f$를 깨뜨리는 알고리즘
% $\bdv$로 환원하는 과정을 제시한다. 현재의 증명에서는 원본 논문 [3]에서와 같은
% 전체 도메인 일방성(full-domain one-wayness)이 아니라, 부분 도메인 일방성
% 하에서의 보안성에만 관심을 둔다. 두 번째 단계에서는 이 환원에서 사용된 복호화
% 오라클 시뮬레이션이 부분 도메인 일방성 하에서 압도적인 확률로 올바르게 동작함을
% 보인다. 이 부분은 원본 증명 [3]과 다르며, 최근 발견된 오류 [15]를 수정한다.
% 마지막으로, 우리는 복호화 오라클 시뮬레이션에 대한 위에서 언급한 분석을 포함하여
% 전체적인 환원의 성공 확률을 분석한다.

% 이 첫 번째 부분에서는 환원이 어떻게 작동하는지를 다시 살펴본다. $\adv$를
% $(\genkey, \enc, \dec)$의 $\indcca$ 공격자로 가정하자. 시간 제한 $\rt$ 내에서,
% $\adv$는 복호화 오라클에 대해 $\qr_{\dec}$개의 질의를 하고, 무작위 오라클 $\hash,
% \gash$에 대해 각각 $\qr_{\hash}, \qr_{\gash}$개의 질의를 수행하며, 특정 확률
% $\negl$보다 높은 능력치로 올바른 평문을 구별해낸다. 이제 환원 $\bdv$을 설명한다.

% \begin{tcolorbox}[colback=white]
% 	\centering
% 	\begin{tabularx}{\linewidth}{CcCcC}
% 		\underline{Challenger $\ch$} & $\xLeftrightarrow{\Exp^{\owfpd}_{f, \sp}}$ & \underline{Adversary $\bdv$} & $\xLeftrightarrow{\Exp^{\indccarom}_{\sch, \sp}}$ & \underline{Adversary $\adv$} \\
% 		\\
% 		$(\pk, \sk) \gets \genkey(1^{\sp})$ & &  &  & \\
% 		\\
% 		$(s^*, t^*) \rgets \set{0, 1}^{\sp - \sp_0} \times \set{0, 1}^{\sp_0}$ \\ $\ct^* \gets \td_{\pk}(s^*, t^*)$ & $\xrightarrow{1^{\sp}, \pk, \ct^*}$ & & & \\
% 		\\
% 		 & & $\orcl_{\hash}, \orcl_{\gash} \gets \set{}, \set{}$ & $\xrightarrow{1^{\sp}, \pk}$ & \\
% 		\\
% 		 & & \multicolumn{3}{c}{\cellcolor{gray!20}$\orc^{\dec_{\sk}}, \orc^{\hash}, \orc^{\gash}$ Query Phase 1} \\
% 		\\
% 		& & & $\xleftarrow{\msg^*_0, \msg^*_1}$ & $\adv$ chooses $\msg^*_0, \msg^*_1 \in \msgspace$ such that $|\msg^*_0| = |\msg^*_1|$ and $\msg^*_0 \neq \msg^*_1$ \\
% 		\\
% 		 & & $\bit \rgets \set{0, 1}, \coin^* \gets \coinspace$ & $\xrightarrow{\ct^*}$ & \\
% 		\\
% 		 & & \multicolumn{3}{c}{\cellcolor{gray!20}$\orc^{\dec_{\sk}}, \orc^{\hash}, \orc^{\gash}$ Query Phase 2} \\
% 		\\
% 		Return $[s' \same s^*]$ & $\xleftarrow{s'}$ & $\bdv$ chooses $s' \in \set{0, 1}^{\sp - \sp_0}$ & $\xleftarrow{\bit'}$ & $\adv$ chooses $\bit' \in \set{0, 1}$ \\
%   \end{tabularx}
% \end{tcolorbox}

% 이 실험에서 세 개의 오라클을 $\bdv$가 처리하기 때문에, 다음을 고려해야한다.
% \begin{itemize}
% 	\item 공격자 $\adv$의 질의에 대해서, 오라클은 유효한 응답을 해야 한다. $\adv$가
% 	오라클이 잘못된 응답을 하고 있다는 것을 감지해서는 안된다.
% 	\item 오라클이 기대하는 확률분포와 일관되어야 한다. 일관되지 않으면 $\adv$가
% 	이상을 감지할 수 있다.
% 	\item 오라클 응답은 일관되어야 한다.
% 	\item 복호화 오라클은 $\bdv$가 비밀키를 모름에도 수행할 수 있어야 한다.
% \end{itemize}

% \textbf{How $\bdv$ simulate $\orc^{\gash}$?}

% \begin{itemize}
% 	\item If $\gamma \in \orcl_{\gash}$, then response $\gash_{\gamma}$
% 	and $\orcl_{\gash} \gets \orcl_{\gash} \cap (\gamma, \gash_{\gamma})$.
% 	\item Otherwise, do following:
% 	\begin{itemize}
% 		\item For some $\delta \in \orcl_{\hash}$, if $c^* = f(\delta, \gamma \xor
% 		\hash_{\delta})$, then $\gash_{\gamma} \gets \delta \xor (\msg_{\bit} \parallel
% 		0^{\sp_1})$.
% 		\item For all $\delta \in \orcl_{\hash}$, if $c^* \neq f(\delta, \gamma \xor
% 		\hash_{\delta})$, then $\gash_{\gamma} \rgets \set{0 ,1}^{\sp}$.
% 		\item response $\gash_{\gamma}$ and $\orcl_{\gash} \gets \orcl_{\gash} \cap
% 		(\gamma, \gash_{\gamma})$.
% 	\end{itemize}
% \end{itemize}

% \textbf{How $\bdv$ simulate $\orc^{\hash}$?}

% \begin{itemize}
% 	\item If $\delta \in \orcl_{\hash}$, then response $\hash_\delta$.
% 	\item Otherwise, response $\hash_{\delta} \rgets \set{0 ,1}^{\sp}$ and
% 	$\orcl_{\hash} \gets \orcl_{\hash} \cap (\delta, \hash_{\delta})$.
% \end{itemize}

% \textbf{How $\bdv$ simulate $\orc^{\dec_{\sk}}$?} 

% \begin{itemize}
% 	\item If $\ct = f(\delta, \hash_{\delta} \xor \gamma)$ and 
% 	$[\gash_{\gamma} \xor \delta]_{\sp_1} = 0^{\sp_1}$, then
% 	response $[\gash_{\gamma} \xor \delta]^n$.
% 	\item Otherwise, response $\reject$.
% \end{itemize}

\end{document}
