\documentclass{article}

\title{RSA-OAEP IND-CCA2 증명}
\author{김동현(wlswudpdlf31@kookmin.ac.kr)}
\date{\today}

\usepackage{style}

% 한국어 지원 패키지
\usepackage{kotex}

% Document setting
\usepackage{geometry}
\geometry{
	a4paper, 
	left=3cm, right=3cm, top=2cm, bottom=2cm, 
	includehead, includefoot}
	\usepackage{fancyhdr} % 머리말과 꼬리말 설정 
\usepackage{lastpage}
\pagestyle{fancy}
\fancyhf{} % 기존 머리말/꼬리말 초기화
\renewcommand{\headrulewidth}{0.4pt} % 머리말 선 두께
\fancyhead[L]{\leftmark} % 왼쪽 머리말
\fancyhead[R]{\rightmark} % 오른쪽 머리말
\renewcommand{\footrulewidth}{0.4pt}
\fancyfoot[L]{FDL}
\fancyfoot[R]{\thepage~/~\pageref{LastPage}} % 가운데 꼬리말에 페이지 번호 추가


% 수학 관련 패키지
\usepackage{amsmath, amssymb}
\usepackage{amsthm}
\usepackage{mathtools} % mathtools 패키지 필요
\newtheorem{theorem}{정리}

% 이미지 관련 패키지
\usepackage{graphicx}

% TikZ 패키지 추가
\usepackage{tikz}

% Table package and setting
\usepackage{tcolorbox}
\usepackage{tabularx}
\usepackage{colortbl} % colortbl 패키지 추가
\newcolumntype{C}{>{\centering\arraybackslash}X}


\begin{document}

\maketitle

\section{기호}

\begin{itemize}
	\item $\sp$ 보안 매개변수
	\item $\pk$ 공개키
	\item $\sk$ 비밀키
	\item $(\pk, \sk) \gets \genkey(1^{\sp})$ 키 생성 알고리즘, 확률론적
	\item $\msg$ 메시지
	\item $\ct$ 암호문
	\item $\ct \gets \enc(\msg, \pk)$ 암호화 함수, 확률론적
	\item $\msg \gets \dec(\ct, \sk)$ 복호화 함수, 결정론적
	\item $\adv$ 공격자
	\item $\coin$ 랜덤 코인
	\item $\msgspace$ 메시지 공간
	\item $\coinspace$ 랜덤 코인 공간
	\item $\bit$ 0 또는 1
	\item $\hash, \gash$ 암호학적 해시 함수
	\item $\mlen$ 메시지 길이
\end{itemize}

\newpage
\section{보안 개념}

\subsection{$\owcpa$}

\begin{tcolorbox}[colback=white]
	\centering
	\begin{tabularx}{\linewidth}{CcC}
		\underline{Challenger $\ch$} & 
		$\xLeftrightarrow{\Exp^{\owcpa}_{\sch, \sp}}$ & 
		\underline{Adversary $\adv$} \\

		 \\

		$(\pk, \sk) \gets \genkey(1^{\sp})$ & 
		$\xrightarrow{1^{\sp}, \pk}$ & 
	   \\

		 \\

		$\msg^* \rgets \msgspace$ \\
		$\ct^* \gets \enc_{\pk}(\msg^*)$ &
		$\xrightarrow{\ct^*}$ &
		 \\

		 \\

		Return $[\msg' \same \msg^*]$ & 
		$\xleftarrow{\msg'}$ & 
		$\adv$ chooses $\msg' \in \msgspace$ \\
  \end{tabularx}
\end{tcolorbox}

% \begin{align}
% 	\begin{split}
% 		\Adv^{\owcpa}_{\adv, \sch}(\sp)
% 		&= \Pr[\Exp^{\owcpa}_{\sch, \sp}(\adv) = 1] \\
% 		&= \underset{\msg^*}{\Pr}[
% 				(\pk, \sk) \gets \genkey(1^{\sp}): 
% 				\adv(\pk, \enc_{\pk}(m^*)) = \msg^*].
% 	\end{split}
% \end{align}

\subsection{\owf}

$f:\set{0, 1}^{\sp} \to \set{0, 1}^{\sp}$ is a trapdoor one-way permutation.

\begin{tcolorbox}[colback=white]
	\centering
	\begin{tabularx}{\linewidth}{CcC}
		\underline{Challenger $\ch$} & 
		$\xLeftrightarrow{\Exp^{\owf}_{f, \sp}}$ & 
		\underline{Adversary $\adv$} \\

		 \\

		& 
		$\xrightarrow{f}$ & 
	     \\

		 \\

		$\msg^* \rgets \msgspace$ \\
		$\ct^* \gets f(\msg^*)$ &
		$\xrightarrow{\ct^*}$ &
		 \\

		 \\

		Return $[\msg' \same \msg^*]$ & 
		$\xleftarrow{\msg'}$ & 
		$\adv$ chooses $\msg' \in \msgspace$ \\
  \end{tabularx}
\end{tcolorbox}

% $\sch$ is $(\rt, \negl)\textsf{-}\owcpa$ secure: \\
% Adversary $\adv$ whose running time is bounded by $\rt$,
% $\Adv^{\owcpa}_{\adv, \sch}(\sp) \le \negl$.

\subsection{\owfpd}

$f: \set{0, 1}^{\sp - \sp_0} \times \set{0, 1}^{\sp_0} \to \set{0, 1}^{\sp -
\sp_0} \times \set{0, 1}^{\sp_0}$ is a trapdoor one-way permutation.

\begin{tcolorbox}[colback=white]
	\centering
	\begin{tabularx}{\linewidth}{CcC}
		\underline{Challenger $\ch$} & 
		$\xLeftrightarrow{\Exp^{\owfpd}_{f, \sp}}$ & 
		\underline{Adversary $\bdv$} \\

		 \\

		 & 
		$\xrightarrow{f}$ & 
	   \\

		 \\

		$(s^*, t^*) \rgets \set{0, 1}^{\sp - \sp_0} \times \set{0, 1}^{\sp_0}$ \\
		$\ct^* \gets f(s^*, t^*)$ &
		$\xrightarrow{\ct^*}$ &
		 \\

		 \\

		Return $[s' \same s^*]$ & 
		$\xleftarrow{s'}$ & 
		$\adv$ chooses $s' \in \set{0, 1}^{\sp - \sp_0}$ \\
  \end{tabularx}
\end{tcolorbox}

% \begin{align}
% 	\begin{split}
% 		\Adv^{\owcpapd}_{\adv, \sch}(\sp)
% 		&= \Pr[\Exp^{\owcpapd}_{f, \sp}(\adv) = 1] \\
% 		&= \underset{s^*, t^*}{\Pr}[
% 				(\pk, \sk) \gets \genkey(1^{\sp}): 
% 				\adv(\pk, \enc_{\pk}(s^*, t^*)) = s^*].
% 	\end{split}
% \end{align}

\subsection{$\owfpds$}

??

\subsection{$\indcca$}

\begin{tcolorbox}[colback=white]
	\centering
	\begin{tabularx}{\linewidth}{CcC}
		\underline{Challenger $\ch$} & 
		$\xLeftrightarrow{\Exp^{\indcca}_{\sch, \sp}}$ & 
		\underline{Adversary $\adv$} \\

		 \\

		$(\pk, \sk) \gets \genkey(1^{\sp})$ & 
		$\xrightarrow{1^{\sp}, \pk}$ & 
	   \\

		 \\

		 \multicolumn{3}{c}{\cellcolor{gray!20}$\orc^{\dec_{\sk}}$ Query Phase 1} \\

		 \\

		 &
		$\xleftarrow{\msg^*_0, \msg^*_1}$ & 
		$\adv$ chooses $\msg^*_0, \msg^*_1 \in \msgspace$ such that 
		$|\msg^*_0| = |\msg^*_1|$ and $\msg^*_0 \neq \msg^*_1$ 	
		 \\

		 \\

		$\bit \rgets \set{0, 1}$ \\
		$\ct^* \gets \enc_{\pk}(\msg^*_{\bit})$ &
		$\xrightarrow{\ct^*}$ &
		 \\

		 \\

		 \multicolumn{3}{c}{\cellcolor{gray!20}$\orc^{\dec_{\sk}}$ Query Phase 2} \\

		 \\

		Return $[\bit' \same \bit]$ & 
		$\xleftarrow{\bit'}$ & 
		$\adv$ chooses $\bit' \in \set{0, 1}$ \\
  \end{tabularx}
\end{tcolorbox}

% \begin{align}
% 	\begin{split}
% 		\Adv^{\indcca}_{\adv, \sch}(\sp)
% 		&= 2 \cdot \Pr[\Exp^{\indcca}_{\sch, \sp}(\adv) = 1] - 1 \\
% 		&= \underset{\bit^*, \coin^*}{\Pr}[
% 				(\pk, \sk) \gets \genkey(1^{\sp});
% 				(\msg^*_0, \msg^*_1) \gets \adv: 
% 				\adv(\pk, \enc_{\pk}(\msg^*_{\bit^*}; \coin^*)) = \bit^*].
% 	\end{split}
% \end{align}

% $\sch$ is $(\rt, \negl)\textsf{-}\indcca$ secure: \\
% Adversary $\adv$ whose running time is bounded by $\rt$,
% $\Adv^{\owcpa}_{\adv, \sch}(\sp) \le \negl$.

\subsection{$\indccarom$}

% \begin{tcolorbox}[colback=white]
% 	\centering
% 	\begin{tabularx}{\linewidth}{CcC}
% 		\underline{Challenger $\ch$} & 
% 		 & 
% 		\underline{Adversary $\adv$} \\

% 		 \\

% 		 $\orcl_{\hash} \gets \set$ &&\\

% 		 \\

% 		& $\xleftarrow{x}$ & $\adv$ chooses $x \in \set{0, 1}^*$ \\

% 		\\

% 		If $(x, y') \in \orcl$ then $y \gets y'$ \\
% 		else $y \rgets \set{0, 1}^{\mlen}$ \\
% 		$\orcl_{\hash} \gets \orcl_{\hash} \cap (x, y)$ &
% 		$\xrightarrow{y}$ & \\

%   \end{tabularx}
% \end{tcolorbox}

\begin{tcolorbox}[colback=white]
	\centering
	\begin{tabularx}{\linewidth}{CcC}
		\underline{Challenger $\ch$} & 
		$\xLeftrightarrow{\Exp^{\indccarom}_{\sch, \sp}}$ & 
		\underline{Adversary $\adv$} \\

		 \\

		$\orcl_{\hash} \gets \set$ \\
		$(\pk, \sk) \gets \genkey(1^{\sp})$ & 
		$\xrightarrow{1^{\sp}, \pk}$ & 
	   \\

		 \\

		 \multicolumn{3}{c}{\cellcolor{gray!20}$\orc^{\hash}, \orc^{\dec_{\sk}}$ Query Phase 1} \\

		 \\

		 &
		$\xleftarrow{\msg^*_0, \msg^*_1}$ & 
		$\adv$ chooses $\msg^*_0, \msg^*_1 \in \msgspace$ such that 
		$|\msg^*_0| = |\msg^*_1|$ and $\msg^*_0 \neq \msg^*_1$ 	
		 \\

		 \\

		$\bit \rgets \set{0, 1}$ \\
		$\ct^* \gets \enc_{\pk}(\msg^*_{\bit})$ &
		$\xrightarrow{\ct^*}$ &
		 \\

		 \\

		 \multicolumn{3}{c}{\cellcolor{gray!20}$\orc^{\hash}, \orc^{\dec_{\sk}}$ Query Phase 2} \\

		 \\

		Return $[\bit' \same \bit]$ & 
		$\xleftarrow{\bit'}$ & 
		$\adv$ chooses $\bit' \in \set{0, 1}$ \\
  \end{tabularx}
\end{tcolorbox}

\newpage
\section{RSA-OAEP}


치환(permutation) $f: \set{0, 1}^{\sp} \To \set{0, 1}^{\sp}$를 다음과 같이
표현한다.
$$
  f: \set{0, 1}^{\mlen + \sp_1} \times \set{0, 1}^{\sp_0}
   \To \set{0, 1}^{\mlen + \sp_1} \times \set{0, 1}^{\sp_0}.
$$
이 때, $\sp = \mlen + \sp_0 + \sp_1$이다.

함수 $f$와 그 역함수 $g$로부터 얻은 OAEP 암호 $(\genkey, \enc, \dec)$를 나타내기
위해, 다음과 같은 두 해시 함수 $\hash, \gash$가 필요하다.
$$
	\hash: \set{0, 1}^{\sp_0} \to \set{0, 1}^{\sp - \sp_0} \quad
	\gash: \set{0, 1}^{\sp - \sp_0} \to \set{0, 1}^{\sp_0}.
$$
OAEP 암호 $(\genkey, \enc, \dec)$는 다음과 같다.
\begin{itemize}
	\item $\genkey(1^{\sp})$: 함수 $f$의 인스턴스 $\pk$, 함수 $g$의 인스턴스
	$\sk$를 생성한다.
	\item $\enc_{\pk}(\msg, \coin)$: $\msg \in \set{0, 1}^{\mlen}$과 $\coin \rgets
	\set{0, 1}^{\sp_0}$가 주어졌을 때, 다음을 계산한다.
	$$
		s = (\msg \parallel 0^{\sp_1}) \xor \gash(\coin) \quad
		t = \coin \xor \hash(s).
	$$
	이후 암호문 $\ct = f(s, t)$를 출력한다.
  \item $\dec_{\sk}(\ct)$: $\sk$를 사용하여 다음을 순서대로 계산할 수 있다.
  \begin{itemize}
    \item $(s,t) = g(\ct)$
    \item $r = t \xor \hash(s), M = s \xor \gash(\coin)$
  \end{itemize}
  만약 $[M]_{\sp_1} = 0^{\sp_1}$이면 $[M]_{\mlen}$을 출력하고, 아니라면
  “$\textsf{Reject}$”를 출력한다. 이 표현에서, $[M]_{k_1}$은 $M$의 하위 $\sp_1$
  비트를, $[M]_{\mlen}$은 $M$의 상위 $\mlen$ 비트를 의미한다.
\end{itemize}

\begin{center}
	\begin{tikzpicture}
		% message box
		\node[draw, rectangle, minimum width=3cm, minimum height=0.6cm] (box) at (1,
		1) {$\msg$}; 
		\draw[<->] ([yshift=0.5cm] box.west) -- ([yshift=0.5cm]
		box.east) node[midway, above] {$n = \sp - \sp_0 - \sp_1$};
		% \draw[<->] ([xshift=-0.5cm] box.north) -- ([xshift=-0.5cm] box.south) node[midway, left] {$h$};
		
		% zero box
		\node[draw, rectangle, minimum width=2cm, minimum height=0.6cm] (box2) at (-1.5,
		1) {$0^{\sp_0}$};

		% seed box
		\node[draw, rectangle, minimum width=2cm, minimum height=0.6cm] (box3) at (-3.5,
		1) {$\coin$};

		% t box
		\node[draw, rectangle, minimum width=2cm, minimum height=0.6cm] (box4) at (-3.5,
		-3.5) {$t$};

		% s box
		\node[draw, rectangle, minimum width=5cm, minimum height=0.6cm] (box5) at (0,
		-3.5) {$s$};

		% hash1 box
		\node[draw, rectangle, minimum width=1cm, minimum height=1cm] (hbox1) at (-1.25,
		-0.5) {$\hash$};

		% hash2 box
		\node[draw, rectangle, minimum width=1cm, minimum height=1cm] (hbox1) at (-1.25,
		-2) {$\gash$};

		% r -> t
		\draw[->] (box3) -- (box4);

	\end{tikzpicture}
\end{center}

\section{OAEP IND-CCA2 증명}

\begin{theorem}
	$\adv$를 OAEP 변환 $(\genkey, \enc, \dec)$에 대해 능력치(advantage)
	$\negl$과 시간(running time) $\rt$를 가지고, 복호화 오라클, 해시 함수
	$\hash$ 및 $\gash$에 각각 $\qr_{\dec}, \qr_{\hash}, \qr_{\gash}$ 회
	질의하는 $\textsf{IND-CCA2}$ 공격자라 하자. 다음을 만족한다.
	$$
    S \ge \frac{1}{\qr_{\hash}} \left(
      \frac{\negl}{2}
       - \frac{2\qr_{\dec}\qr_{\gash} + \qr_{\dec} + \qr_{\gash}}{2^{\sp_0}}
       - \frac{2\qr_{\dec}}{2^{\sp_1}}
    \right).
	$$
  이 때, $\rt' \le t \cdot \qr_{\hash} \cdot \qr_{\gash} \cdot (\comp{f} +
   \bigo(1))$이고, $\comp{f}$는 함수 $f$의 시간 복잡도를 의미한다.
\end{theorem}



우리는 보조정리 2를 세 단계로 증명한다. 첫 번째 단계에서는 $\indcca$ 적대자
$\adv$를 부분 도메인 일방성(partial-domain one-wayness) $f$를 깨뜨리는 알고리즘
$\bdv$로 환원하는 과정을 제시한다. 현재의 증명에서는 원본 논문 [3]에서와 같은
전체 도메인 일방성(full-domain one-wayness)이 아니라, 부분 도메인 일방성
하에서의 보안성에만 관심을 둔다. 두 번째 단계에서는 이 환원에서 사용된 복호화
오라클 시뮬레이션이 부분 도메인 일방성 하에서 압도적인 확률로 올바르게 동작함을
보인다. 이 부분은 원본 증명 [3]과 다르며, 최근 발견된 오류 [15]를 수정한다.
마지막으로, 우리는 복호화 오라클 시뮬레이션에 대한 위에서 언급한 분석을 포함하여
전체적인 환원의 성공 확률을 분석한다.

이 첫 번째 부분에서는 환원이 어떻게 작동하는지를 다시 살펴본다. $\adv$를
$(\genkey, \enc, \dec)$의 $\indcca$ 공격자로 가정하자. 시간 제한 $\rt$ 내에서,
$\adv$는 복호화 오라클에 대해 $\qr_{\dec}$개의 질의를 하고, 무작위 오라클 $\hash,
\gash$에 대해 각각 $\qr_{\hash}, \qr_{\gash}$개의 질의를 수행하며, 특정 확률
$\negl$보다 높은 능력치로 올바른 평문을 구별해낸다. 이제 환원 $\bdv$을 설명한다.

\begin{tcolorbox}[colback=white]
	\centering
	\begin{tabularx}{\linewidth}{CcCcC}
		\underline{Challenger $\ch$} & 
		$\xLeftrightarrow{\Exp^{\owfpd}_{f, \sp}}$ & 
		\underline{Adversary $\bdv$} &
		$\xLeftrightarrow{\Exp^{\indccarom}_{\sch, \sp}}$ &
		\underline{Adversary $\adv$} \\

		 \\

		 & 
		$\xrightarrow{f}$ &  &  &
	     \\

		 \\

		$(s^*, t^*) \rgets \set{0, 1}^{\sp - \sp_0} \times \set{0, 1}^{\sp_0}$ \\
		$\ct^* \gets f(s^*, t^*)$ &
		$\xrightarrow{\ct^*}$ & & &
		 \\

		 \\

		 & & $\orcl_{\hash}, \orcl_{\gash} \gets \set{}, \set{}$ & $\xrightarrow{1^{\sp}, \pk}$ & \\

		 \\

		 & & \multicolumn{3}{c}{\cellcolor{gray!20}$\orc^{\dec_{\sk}}, \orc^{\hash}, \orc^{\gash}$ Query Phase 1} \\

		 \\

		& & & 
		$\xleftarrow{\msg^*_0, \msg^*_1}$ & 
		$\adv$ chooses $\msg^*_0, \msg^*_1 \in \msgspace$ such that 
		$|\msg^*_0| = |\msg^*_1|$ and $\msg^*_0 \neq \msg^*_1$ \\

		 \\

		 & & & $\xrightarrow{\ct^*}$ & \\

		 \\

		 & & \multicolumn{3}{c}{\cellcolor{gray!20}$\orc^{\dec_{\sk}}, \orc^{\hash}, \orc^{\gash}$ Query Phase 2} \\

		 \\

		 \\

		Return $[s' \same s^*]$ & 
		$\xleftarrow{s'}$ & 
		$\bdv$ chooses $s' \in \set{0, 1}^{\sp - \sp_0}$ &
		$\xleftarrow{\bit'}$ &
		$\adv$ chooses $\bit' \in \set{0, 1}$ \\
  \end{tabularx}
\end{tcolorbox}

% 이 실험에서 세 개의 오라클을 $\bdv$가 처리하기 때문에, 다음을 고려해야한다.
% \begin{itemize}
% 	\item 공격자 $\adv$의 질의에 대해서, 오라클은 유효한 응답을 해야 한다. $\adv$가
% 	오라클이 잘못된 응답을 하고 있다는 것을 감지해서는 안된다.
% 	\item 오라클이 기대하는 확률분포와 일관되어야 한다. 일관되지 않으면 $\adv$가
% 	이상을 감지할 수 있다.
% 	\item 오라클 응답은 일관되어야 한다.
% 	\item 복호화 오라클은 $\bdv$가 비밀키를 모름에도 수행할 수 있어야 한다.
% \end{itemize}

\textbf{How $\bdv$ simulate $\orc^{\gash}$?}

\begin{itemize}
	\item If $\gamma \in \orcl_{\gash}$, then response $\gash_{\gamma}$
	and $\orcl_{\gash} \gets \orcl_{\gash} \cap (\gamma, \gash_{\gamma})$.
	\item Otherwise, do following:
	\begin{itemize}
		\item For some $\delta \in \orcl_{\hash}$, if $c^* = f(\delta, \gamma \xor
		\hash_{\delta})$, then $\gash_{\gamma} \gets \delta \xor (\msg_{\bit} \parallel
		0^{\sp_1})$.
		\item For all $\delta \in \orcl_{\hash}$, if $c^* \neq f(\delta, \gamma \xor
		\hash_{\delta})$, then $\gash_{\gamma} \rgets \set{0 ,1}^{\sp}$.
		\item response $\gash_{\gamma}$ and $\orcl_{\gash} \gets \orcl_{\gash} \cap
		(\gamma, \gash_{\gamma})$.
	\end{itemize}
\end{itemize}

\textbf{How $\bdv$ simulate $\orc^{\hash}$?}

\begin{itemize}
	\item If $\delta \in \orcl_{\hash}$, then response $\hash_\delta$.
	\item Otherwise, response $\hash_{\delta} \rgets \set{0 ,1}^{\sp}$ and
	$\orcl_{\hash} \gets \orcl_{\hash} \cap (\delta, \hash_{\delta})$.
\end{itemize}

\textbf{How $\bdv$ simulate $\orc^{\dec_{\sk}}$?} 

\begin{itemize}
	\item If $\ct = f(\delta, \hash_{\delta} \xor \gamma)$ and 
	$[\gash_{\gamma} \xor \delta]_{\sp_1} = 0^{\sp_1}$, then
	response $[\gash_{\gamma} \xor \delta]^n$.
	\item Otherwise, response $\reject$.
\end{itemize}

\end{document}
