\section{논문정보}

\begin{itemize}
	\item 제목: RSA-OAEP is Secure under the RSA Assumption
	\item 저자: Eiichiro Fujisaki1, Tatsuaki Okamoto, David Pointcheval, and Jacques Stern
	\item 년도: 2001년
	\item 초록: 최근 Victor Shoup은 적응적 선택 암호문 공격에 대한 OAEP의 보안성에 관한 널리
	받아들여진 결과에 틈이 있음을 지적하였다. 더욱이, 그는 기본 트랩도어 치환의
	단방향성만으로는 OAEP의 보안성을 증명할 수 없을 것으로 예상된다는 점을 보였다.
	본 논문은 OAEP의 보안성에 대한 또 다른 결과를 제시한다. 즉, 본 논문에서는 무작위
	오라클 모델에서, 기본 치환의 부분 영역 단방향성(partial-domain one-wayness)
	하에서, OAEP가 적응적 선택 암호문 공격에 대해 의미론적 보안성을 제공함을
	증명한다. 따라서, 이는 형식적으로 더 강한 가정을 사용한다. 그럼에도 불구하고,
	RSA 함수의 부분 영역 단방향성이 (전체 영역) 단방향성과 동치이므로, RSA-OAEP의
	보안성은 단순한 RSA 가정만으로도 증명될 수 있음을 알 수 있다. 다만, 그
	축소(reduction)는 타이트하지 않다.
\end{itemize}

\begin{itemize}
	\item 2장에서는 증명에 필요한 보안 개념을 다룬다.
	\item 3장에서는 RSA-OAEP에 대한 설명을 다룬다.
	\item 4장과 5장에서는 RSA-OAEP가 IND-CCA2 보안을 만족함을 다룬다. 4장은
	Crypto2001에 나온 논문에 대한 증명을, 5장은 Journal of Cryptology에 나온
	논문에 대한 증명을 다룬다. 4장은 증명을 마무리 하기는 했으나, 빈틈이
	매우많으므로, 4장 대신 5장을 확인한다.
\end{itemize}