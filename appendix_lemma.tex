\section{보조정리}

\begin{tcolorbox}[colback=white]
    \begin{lemma}
        $E_1$, $E_2$, $F_1$, $F_2$를 하나의 확률 공간 상에 정의된 사건들이라고
        하자. 만약
        $$
            \Pr[E_1 \land \neg F_1] = \Pr[E_2 \land \neg F_2], \quad \Pr[F_1] = \Pr[F_2] = \varepsilon
        $$
        가 성립한다면, 다음을 만족한다.
        $$
            |\Pr[E_1] - \Pr[E_2]| \leq \varepsilon
        $$
    \end{lemma}
\end{tcolorbox}

\begin{proof}
    $|\Pr[E_1] - \Pr[E_2]|$은 다음과 같이 표현 가능하다.
    $$
        |\Pr[E_1 \land \neg F_1] + \Pr[E_1 \land F_1] - \Pr[E_2 \land \neg F_2] - \Pr[E_2 \land F_2]|.
    $$
    가정에 의해 $\Pr[E_1 \land \neg F_1] = \Pr[E_2 \land \neg F_2]$이므로, 다음과 같이 식을 줄일 수 있다.
    $$
        |\Pr[E_1 \land F_1] - \Pr[E_2 \land F_2]|.
    $$
    위 식을 조건부 확률로 표현하면 다음과 같다.
    $$
        |\Pr[E_1 \mid F_1] \cdot \Pr[F_1] - \Pr[E_2 \mid F_2] \cdot \Pr[F_2]|.
    $$
    가정에 의해 $\Pr[F_1] = \Pr[F_2] = \varepsilon$이므로, 위 식은 다음과 같이
    표현할 수 있다.
    $$
    \varepsilon \cdot |\Pr[E_1 \mid F_1]- \Pr[E_2 \mid F_2]|.
    $$
    어떤 사건의 조건부 확률은 언제나 0 이상 1 이하이므로, $|\Pr[E_1 \mid F_1] -
    \Pr[E_2 \mid F_2]| \leq 1$을 만족한다. 따라서 다음을 만족한다.
    $$
    |\Pr[E_1] - \Pr[E_2]| = \varepsilon \cdot |\Pr[E_1 \mid F_1] - \Pr[E_2 \mid F_2]| \leq \varepsilon \cdot 1 = \varepsilon.
    $$
\end{proof}

\begin{memo}
    두 사건 $E_1, E_2$의 확률이 어떤 Bad 사건 $F$가 발생할 때만 차이가
    발생한다면, 두 사건의 확률 차이는 Bad 사건의 확률 이하로 조절된다는
    의미이다. 암호학에서는, 두 게임 간 차이를 분석할 때 사용할 수 있다. 예를 들어, 
    \begin{itemize}
        \item $S_1$: $\GAME_1$에서의 성공 사건
        \item $S_2$: $\GAME_2$에서의 성공 사건
        \item $\varepsilon$: 각각의 게임에서 발생할 수 있는 Bad 사건의 확률
    \end{itemize}
    이라고 할 때, 두 게임이 Bad 사건 외에서는 동일하게 동작하면, 두 게임의 성공
    확률 차이는 Bad 사건의 확률 이하로 상한된다. 즉, $|\Pr[S_1] - \Pr[S_2]| \leq \varepsilon$이다.
\end{memo}