\section{보조정리}

\subsection{Lemma A}

\begin{tcolorbox}[colback=white, sharp corners, boxrule=0.7pt]
    \begin{lemma}
        $E_1$, $E_2$, $F_1$, $F_2$를 하나의 확률 공간 상에 정의된 사건들이라고
        하자. 만약
        $$
            \Pr[E_1 \land \neg F_1] = \Pr[E_2 \land \neg F_2], \quad \Pr[F_1] = \Pr[F_2] = \varepsilon
        $$
        가 성립한다면, 다음을 만족한다.
        $$
            |\Pr[E_1] - \Pr[E_2]| \leq \varepsilon
        $$
    \end{lemma}
\end{tcolorbox}

\begin{proof}
    $|\Pr[E_1] - \Pr[E_2]|$은 다음과 같이 표현 가능하다.
    $$
        |\Pr[E_1 \land \neg F_1] + \Pr[E_1 \land F_1] - \Pr[E_2 \land \neg F_2] - \Pr[E_2 \land F_2]|.
    $$
    가정에 의해 $\Pr[E_1 \land \neg F_1] = \Pr[E_2 \land \neg F_2]$이므로, 다음과 같이 식을 줄일 수 있다.
    $$
        |\Pr[E_1 \land F_1] - \Pr[E_2 \land F_2]|.
    $$
    위 식을 조건부 확률로 표현하면 다음과 같다.
    $$
        |\Pr[E_1 \mid F_1] \cdot \Pr[F_1] - \Pr[E_2 \mid F_2] \cdot \Pr[F_2]|.
    $$
    가정에 의해 $\Pr[F_1] = \Pr[F_2] = \varepsilon$이므로, 위 식은 다음과 같이
    표현할 수 있다.
    $$
    \varepsilon \cdot |\Pr[E_1 \mid F_1]- \Pr[E_2 \mid F_2]|.
    $$
    어떤 사건의 조건부 확률은 언제나 0 이상 1 이하이므로, $|\Pr[E_1 \mid F_1] -
    \Pr[E_2 \mid F_2]| \leq 1$을 만족한다. 따라서 다음을 만족한다.
    $$
    |\Pr[E_1] - \Pr[E_2]| = \varepsilon \cdot |\Pr[E_1 \mid F_1] - \Pr[E_2 \mid F_2]| \leq \varepsilon \cdot 1 = \varepsilon.
    $$
\end{proof}

\begin{memo}
    두 사건 $E_1, E_2$의 확률이 어떤 Bad 사건 $F$가 발생할 때만 차이가
    발생한다면, 두 사건의 확률 차이는 Bad 사건의 확률 이하로 조절된다는
    의미이다. 암호학에서는, 두 게임 간 차이를 분석할 때 사용할 수 있다. 예를 들어, 
    \begin{itemize}
        \item $S_1$: $\GAME_1$에서의 성공 사건
        \item $S_2$: $\GAME_2$에서의 성공 사건
        \item $\varepsilon$: 각각의 게임에서 발생할 수 있는 Bad 사건의 확률
    \end{itemize}
    이라고 할 때, 두 게임이 Bad 사건 외에서는 동일하게 동작하면, 두 게임의 성공
    확률 차이는 Bad 사건의 확률 이하로 상한된다. 즉, $|\Pr[S_1] - \Pr[S_2]| \leq \varepsilon$이다.
\end{memo}

\subsection{Lemma B}

증명은 지금 이해하기에는 너무 어려우니, 결과만 알도록 하자.

\begin{tcolorbox}[colback=white, sharp corners, boxrule=0.7pt]
    \begin{lemma}
        다음과 같은 방정식을 고려하자.
        $$
            t + \alpha u \equiv c \pmod{N}.
        $$
        이 방정식은 $t$와 $u$에 대해 해를 가지며, 이때 $t$와 $u$는 모두
        $2^{\lambda_0}$보다 작은 값을 가진다고 하자.

        $\alpha \in [0, N-1]$인 모든 값들에 대해, $2^{2\lambda_0 +
        6}/N$의 비율을 제외하고는, $(t, u)$는 유일하게 결정되며, 그 해는
        $O((\log N)^3)$ 시간 내에 계산될 수 있다.
    \end{lemma}
    \label{lem:small-solution}
\end{tcolorbox}

\begin{proof}
    다음과 같은 격자 $L(\alpha)$를 고려한다.
    $$
        L(\alpha) = \{(x, y) \in \mathbb{Z}^2 \mid x - \alpha y \equiv 0 \pmod{N} \}.
    $$
    우리는 $L(\alpha)$가 $\ell$-good 격자(그리고 $\alpha$가 $\ell$-good 값)라고
    부른다. 그 의미는 $L(\alpha)$ 안에 유클리드 노름 기준으로 길이가 $\ell$
    이하인 $0$이 아닌 벡터가 존재하지 않을 경우이다. 그렇지 않은 경우에는
    $\ell$-bad 격자(그리고 $\ell$-bad 값)라고 부른다.$\ell$-bad 격자의 수는 대략
    $\pi \ell^2$보다 작으며, 우리는 이를 $4\ell^2$로 상계한다. 실제로,
    $\alpha$에 대한 각 bad 값은 반지름 $\ell$인 원판 안의 정수 좌표 점에
    대응된다. 또한, 위 격자들은 서로 교차하는 지점이 $(0, 0)$ 하나뿐인데, 이는
    $\ell < p$일 때 성립하며, 여기서 $p$는 $N$의 가장 작은 소인수이다. 따라서
    $\alpha$에 대한 bad 값의 비율은 $4\ell^2 / N$보다 작다.

    $\ell$-good 격자 $L(\alpha)$가 주어졌다고 하자. 이 경우, 가우스 축소
    알고리듬(Gaussian reduction algorithm)을 적용할 수 있으며, 시간 복잡도
    $\mathcal{O}((\log N)^3)$ 내에 두 개의 0이 아닌 벡터 $U$와 $V$로 구성된
    $L(\alpha)$의 기저를 얻을 수 있다.  
    이때 다음 조건을 만족한다:
    $$
        \|U\| \leq \|V\| \quad \text{and} \quad |(U, V)| \leq \|U\|^2 / 2.
    $$

    점 $T = (t, u)$를 다음과 같이 정의하자. 여기서 $(t, u)$는 방정식 $t + \alpha u \equiv c \pmod{N}$의 해이며, 
    $t$와 $u$는 모두 $2^{k_0}$보다 작다. 이때 $T$는 다음과 같이 표현된다:  
    $$
        T = \lambda U + \mu V
    $$
    여기서 $\lambda$, $\mu$는 어떤 실수(real)이다. 이때 다음 부등식이 성립한다:
    $$
    \|T\|^2 = \lambda^2 \|U\|^2 + \mu^2 \|V\|^2 + 2\lambda \mu (U, V) 
    \geq (\lambda^2 + \mu^2 - \lambda \mu) \cdot \|U\|^2
    $$
    $$
    \geq \left((\lambda - \mu/2)^2 + \frac{3\mu^2}{4}\right) \cdot \|U\|^2 
    \geq \frac{3\mu^2}{4} \cdot \|U\|^2 
    \geq \frac{3\mu^2 \ell^2}{4}.
    $$
    게다가 우리는 $\|T\|^2 \leq 2 \cdot 2^{2k_0}$임을 알고 있다. 따라서 다음
    부등식이 성립한다:
    \[
    |\mu| \leq \frac{2\sqrt{2} \cdot 2^{k_0}}{\sqrt{3} \cdot \ell}
    \quad \text{and} \quad
    |\lambda| \leq \frac{2\sqrt{2} \cdot 2^{k_0}}{\sqrt{3} \cdot \ell}
    \quad \text{(대칭성에 의해)}.
    \]
    처음부터 $\ell = 2^{k_0 + 2} > 2^{k_0 + 2} \sqrt{\frac{2}{3}}$로 설정했다고
    가정하면, 다음이 성립한다:
    \[
    -\frac{1}{2} < \lambda, \qquad \mu < \frac{1}{2}.
    \]

    방정식의 임의의 정수 해 $T_0 = (t_0, u_0)$를 선택하자.  
    이를 위해 $u_0$를 임의의 정수로 선택하고 $t_0 = c - \alpha u_0 \bmod N$으로 정의한다.  

    이제 $T_0$를 기저 $(U, V)$에 대해 실수 좌표 $\rho$, $\sigma$로 표현하면  
    \[
    T_0 = \rho U + \sigma V
    \]
    이다. 이러한 좌표 $(\rho, \sigma)$는 계산이 가능하며, $T - T_0$는 동차 방정식의 해이므로  
    이는 격자점에 해당하며 다음과 같이 표현된다:
    \[
    T - T_0 = aU + bV,
    \]
    여기서 $a$, $b$는 미지의 정수이다.

    하지만 우리는 또한 다음을 만족함을 알고 있다:
    \[
    T = T_0 + aU + bV = (\rho + a)U + (\sigma + b)V = \lambda U + \mu V,
    \]
    여기서
    \[
    -\frac{1}{2} \leq \lambda, \mu \leq \frac{1}{2}.
    \]

    \paragraph{결론.} 
    따라서 $a$와 $b$는 각각 $-\rho$와 $-\sigma$에 가장 가까운 정수이다.  
    $a$, $b$, $\rho$, $\sigma$가 주어지면 $\lambda$, $\mu$를 쉽게 복원할 수 있으며,  
    그로부터 $t$, $u$를 계산할 수 있다. 이 값들은 반드시 유일하다.
\end{proof}